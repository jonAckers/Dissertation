% Settings file
%
% This template is based on the UiB PhD thesis template
%

\include{settings}	% Document settings file

%
% We use the book class
%

\documentclass[12pt,openany]{book}

%
% The geometry package allows consistent control of page layout
%

\usepackage[paper = a4paper,         %To be shrunk to 80% when printed
			twoside,                 %Two-side mode, switches margins on 
			bindingoffset = 2mm,     %Offset for binding side of page
			hmargin = 25mm,          %Left and right margin
			vmargin = 25mm,          %Top and bottom margin
			dvipdfm]
			{geometry}


%
% Font settings
%
\usepackage[T1]{fontenc}	% Activate Type 1 fonts
% \usepackage{mathptmx}       % Use Times font, also for math
\usepackage[scaled]{helvet} % for sans serif fonts (\textsf{...} or \sffamiliy)
\usepackage{sectsty}        % Change section and chapter header
\allsectionsfont{\usefont{T1}{phv}{bc}{n}\selectfont} % Set chapter/section header to narrow helvectica (arial-like)
%\usepackage[scaled]{luximono} % for monospaced fonts (\texttt{...} or \ttfamily)


%
% Packages we need
%
\usepackage{graphicx}       % Handles figures
\usepackage[utf8]{inputenc} % We want æøå
\usepackage{amsmath}
\usepackage{amsfonts}
% \usepackage[mathcal]{euscript} % For calligraphy fonts
\usepackage{booktabs}       % Publication quality tables
\usepackage{setspace}       % Easy setting of line spacing
\usepackage{forloop}		% For-loops!
% \usepackage{float}
\usepackage[dvipsnames]{color}   % named colors
\usepackage[procnames]{listings} % beautiful listings
\usepackage{units}				 % semantically represent numbers with units 
\usepackage[table]{xcolor}
\usepackage[final]{pdfpages}
\usepackage[backend=bibtex,maxbibnames=99]{biblatex}
\addbibresource{Bibliography.bib}
\usepackage{epigraph}			%for quotation
\setlength\epigraphwidth{8cm}
%\setlength\epigraphrule{0pt} %remove line in quote
\usepackage{textcomp}
\usepackage{sidecap} %for figure caption on the side
\usepackage{caption}
\usepackage{subcaption}
\usepackage{pifont}
\newcommand{\cmark}{\ding{51}}%
\newcommand{\xmark}{\ding{55}}
\usepackage[inline]{enumitem}
\usepackage[hidelinks]{hyperref}
\usepackage{amssymb}
\usepackage{multirow}
\usepackage{siunitx}
\usepackage{pgf-umlcd}
\renewcommand{\umldrawcolor}{black}
\usepackage{xcolor, soul}
\usepackage{tcolorbox}
\usepackage{dirtree}
\definecolor{ckksRed}{rgb}{0.9568627451,0.5921568627,0.5568627451}
\definecolor{mekksBlue}{rgb}{0.4862745098,0.8549019608,0.7960784314}
\definecolor{homomorphicGreen}{rgb}{0.7921568627,1.,0.7490196078}
\definecolor{abstractYellow}{rgb}{0.9921568627,0.9490196078,0.8}

\usepackage{tikz}
\usetikzlibrary{shapes.geometric,arrows,shadows,positioning,patterns}

\newlength{\upBranch} % shift up the text  lines <<<<
\setlength{\upBranch}{0.5ex} % 

\newlength{\tolineSpace} % blank space bellow text  lines  <<<
\setlength{\tolineSpace}{1mm}% 

\usepackage{floatrow}

\usepackage{xpatch} % needed <<<<<<<<
\makeatletter

\xpatchcmd{\dirtree} % root
{\vbox{\@nameuse{DT@body@1}}}
{\raisebox{-\tolineSpace}{\vbox{\@nameuse{DT@body@1}}}}
{}{}    

\xpatchcmd{\dirtree} % below space
{\advance\dimen\z@ by-\@nameuse{DT@lastlevel@\the\DT@countiv}\relax}
{\advance\dimen\z@ by-\tolineSpace \advance\dimen\z@ by-\@nameuse{DT@lastlevel@\the\DT@countiv}\relax}
{}{}
    
\xpatchcmd{\dirtree}% shift up the text  lines
{\kern\DT@sep\box\z@\endgraf}
{\kern\DT@sep\raisebox{-\upBranch}{\box\z@}\endgraf}
{}{} 

\newcommand\myheading[1]{\hspace{-1em}\textbf{#1}\\}

\usepackage{titlesec} %fix header fonts
\titleformat{\title}[display]
  {\normalfont\rmfamily\huge\bfseries\color{black}}
  {\chaptertitlename\ \thechapter}{20pt}{\Huge}
\titleformat{\chapter}[display]
  {\normalfont\rmfamily\huge\bfseries\color{black}}
  {\chaptertitlename\ \thechapter}{20pt}{\Huge}
\titleformat{\section}
  {\normalfont\rmfamily\Large\bfseries\color{black}}
  {\thesection}{1em}{}
\titleformat{\subsection}
  {\normalfont\rmfamily\Large\bfseries\color{black}}
  {\thesubsection}{1em}{}
\titleformat{\subsubsection}
{\normalfont\rmfamily\bfseries\color{black}}
{\thesubsubsection}{1em}{\quad}


%
% Set fancy page header using fancyhdr package
% 
\usepackage{fancyhdr}
\pagestyle{fancy}

% Ensure that the chapter and section headings are in lowercase
\setlength{\headheight}{15pt}
\renewcommand{\chaptermark}[1]{\markboth{#1}{}}
\renewcommand{\sectionmark}[1]{\markright{\thesection\ #1}}

% Delete current section for header/footer
\fancyhf{}

% Define header/footer layout
\fancyhead[LE,RO]{\bfseries\thepage}
\fancyhead[LO]{\bfseries\rightmark}
\fancyhead[RE]{\bfseries\leftmark}
\renewcommand{\headrulewidth}{0.5pt}

% make space for the rule
\fancypagestyle{plain}{
	\fancyhead{} %get rid of the headers on plain pages
	\renewcommand{\headrulewidth}{0pt} % and the line
}
\usepackage[font={small,it,rm}]{caption}
\usepackage{atveryend}
%\usepackage{thumbs}


%
% Set header height to 30pt to make room for two rows in the header
% (useful if you have very long chapter names)
%
%\headheight 30pt


%
% Include user-defined macros
%
\include{macros}

%
% TeX is very proud of its hyphenation engine, to the point where it 
% hyphenates _everything_ just to show off.  Increasing penalty for 
% hyphenation, and increase tolerance for underfull boxes can make 
% the text easier to read, at the expense of making the text less aligned
% to the right margin
%
\hyphenpenalty=10
\tolerance=1000


% Redefine include command -> input command
%\renewcommand{\include}{\input}

% Include only these chapters
%\includeonly{introduction}
\DefineBibliographyStrings{english}{%
  bibliography = {References},
}


%
% Set up the frontpage, with author, title, etc
%
%{\fontsize{28}{30}\usefont{OT1}{phv}{bc}{n}\selectfont 
%{\fontsize{28}{30}\usefont{T1}{phv}{bc}{n}\selectfont Exchange of water masses
\title{{\fontsize{25}{24}\sffamily\textbf{Privacy-Preserving\\Moving Object Detection}}
	\author{
	\large\textbf{Computer Science Tripos - Part II}
	    \vspace{2em}\\
	    \large\textbf{\textit{Candidate Number}}
	    \vspace{4em}\\
		\includegraphics[width=74mm]{figures/cambridge}\vspace{4em}\\
		Department of Computer Science and Technology \\
		University of Cambridge}
	\huge\date{Friday 13 May, 2022}
}


%====================================================
%------------------ BEGIN DOCUMENT ------------------
%====================================================
\begin{document}
%Cause all references in bibtex file to appear in the 'References' section, even
%if they are not explicitly cite'ed in the document
%\nocite{*}

\renewcommand{\familydefault}{\sfdefault} %KD

%--------------------------------------------------------------------------------------------------
% FRONT MATTER
%--------------------------------------------------------------------------------------------------
\maketitle
\frontmatter
\normalfont\rmfamily
\pagestyle{plain}

\chapter*{Declaration of Originality}

\indent \indent
I, Jonathon Ackers of Girton College, being a candidate for Part II of the Computer Science Tripos, hereby declare that this dissertation and the work described in it are my own work, unaided except as may be specified below, and that the dissertation does not contain material that has already been used to any substantial extent for a comparable purpose. I am content for my dissertation to be made available to the students and staff of the University.

\bigskip \bigskip

Signed: Jonathon Ackers

\bigskip \bigskip

Date: 13\textsuperscript{th} May 2022

\pagestyle{plain}

\chapter*{Proforma}
\ldots PROFORMA \ldots

\tableofcontents
\listoffigures
\listoftables

%--------------------------------------------------------------------------------------------------
% MAIN MATTER
%--------------------------------------------------------------------------------------------------
\mainmatter

%
% Main chapters
%
% \pagenumbering{arabic}
% \setcounter{page}{1}

\chapter{Introduction}
\label{chap:introduction}

\section{Motivation}
\label{sec:motivation}
\indent \indent
Computers have improved almost every aspect of modern life. Recently, home security has become a target of the technology revolution. Companies like Ring \cite{RING} and Eufy \cite{EUFY} offer IoT devices like doorbells and cameras to allow their customers to continuously monitor their property. On top of traditional surveillance, these companies also provide software solutions to analyse footage. For example, a doorbell may recognise a visitor or alert to the presence of a stranger. However, the computational intensity of inference means footage must be transferred to more powerful servers. 
\smallskip \\ \indent
To provide security, video is encrypted before transmission to the server. However, the footage must be decrypted when inference is being performed. This poses significant privacy concerns. Decrypting footage on the server exposes the opportunity for employees of these companies to access video content. Consequently, malicious actors could use this information to monitor peoples' location, appraise their belongings, extort subjects, and more. Similar risks exist if the company is hacked and raw video data is exfiltrated.  Homomorphic Encryption (henceforth HE) may provide a solution to this.
\smallskip \\ \indent
In cryptography, HE describes encryption schemes that allow mathematical operations to be performed directly on encrypted data, or \textit{ciphertext}, rather than on raw data, or \textit{plaintext}. For example, consider the calculation $3 \times 5$. In a traditional encryption scheme, the plain values $3$ and $5$ would be multiplied and then encrypted. Using a homomorphic scheme, $3$ and $5$ can be encrypted, and the ciphertexts multiplied, resulting in a ciphertext that produces $15$ when decrypted. However, HE is a developing research area, so has limitations. HE ciphertexts are much larger than unencrypted data, so operations' time and space complexity significantly increased. Similarly, not all operations are available in the HE domain, and those that are varies between schemes, so the choice of scheme is critical to success. An open question is, can this technique be scaled to more complex algorithms, like those required for surveillance?
\smallskip \\ \indent
More specifically, this project aims to investigate if it is possible to extract moving objects from HE video data. Moving object detection is fundamental to surveillance. Detecting when, for example, somebody enters a property allows security systems to alert their owners, possibly pre-empting a break-in. However, more than just motion must be sensed. Objects must be extracted and analysed to prevent users from being notified of unimportant events like, for example, leaves blowing onto a property. Gradual changes and random noise in an environment make modelling a background for object detection a significant challenge to overcome.






\section{Related Work}
\label{sec:relatedWork}
\indent \indent
There have been many attempts to improve video inference privacy using malleable encryption, but none are without flaws. In 2013, Chu et al.\ \cite{Chu} proposed an encryption scheme that supports real-time moving object detection. However, this was quickly shown to suffer from information leakage, leaving it vulnerable to chosen-plaintext attacks\footnote{A \textit{chosen-plaintext attack} is a scenario in which an adversary can freely encrypt plaintexts of their choosing and analyse the resulting ciphertexts.}. Similarly, in 2017, Lin et al.\ \cite{Lin} proposed an encryption scheme to achieve the same goal by only encrypting some of the bits in each pixel, but this is unprotected against steganographic\footnote{\textit{Steganography} describes the technique of securing messages through information hiding. Unlike cryptography, where the existence of a message is known, but its contents are not, steganography attempts to hide the message's existence.} attacks. Therefore, while research has solved the weaknesses in privacy, it is yet to offer a solution that also preserves security, removing utility to real-world applications.
\smallskip \\ \indent
Likewise, researchers have been investigating inference using HE for many years. In 2012, Graepel et al.\ \cite{Graepel} introduced machine learning in the HE domain. Dowlin et al.\ \cite{Dowlin} built upon this when they developed the CryptoNets model for deep learning with HE in 2016. However, deep learning neural networks are considered overly complex for moving-object detection. Instead, Gaussian Mixture Models (GMMs) are the most common technique for background modelling. There is much less HE research into this area of unsupervised learning. The best example comes from 2013 when Pathak and Raj \cite{Pathak} proposed a HE implementation of a GMM for audio inference. While this work can be used to establish a foundation for GMM implementation in the HE domain, audio and video analysis details differ, so its relevance is bounded. There do not seem to be any investigations linking HE and GMMs to video analysis.
\smallskip \\ \indent
The most prevailing explanation for this lack of research is the limited applicability of HE to real-time applications due to high computational complexity. However, a consequence of this is that the usefulness of HE in video inference is not well documented. Moreover, as computing capability and hardware acceleration advance, the relative difficulty of HE operations will reduce. Therefore, more insight into the applicability will become increasingly valuable, as suggested by the growing popularity of HE research. This project attempts to offer some understanding of this field by investigating HE for surveillance through techniques to optimise network transmission and modification of standard inference algorithms to support HE data, overcoming the inherent computational challenges.






\section{Threat Model}
\label{sec:threatModel}
\indent \indent
The Machine Learning as a Service (MLaaS) framework describes a business model in which customers send data to a server for machine learning inference to be performed; then, results are returned. Specifically for this project, security companies analyse video surveillance data remotely. Suppose that a subscriber to one of these services, \textit{Alice}, uses a camera to record activity at her front door. This exposes two critical threats: 
\begin{enumerate*}[label=$(\roman*)$]
    \item an adversary, \textit{Eve}, may eavesdrop on the transmitted data while it is in flight, and
    \item a malicious actor, \textit{Mallory}, may exfiltrate the data while the server stores it, either by hacking the company or or by having privileged access, revealing a range of privacy risks including identity theft, monitoring intimate behaviours of household members, identifying household objects, and more.
\end{enumerate*}
Figure \ref{fig:threatModel} illustrates this succinctly. The first threat can be mitigated relatively easily using cryptographic protocols such as TLS ~\cite{TLS}. However, the second threat is much more difficult to defend against, particularly because data must usually be decrypted before inference ~\cite{Bae}. 
\smallskip \\ \indent
Fortunately, HE offers a potential solution to both risks. Firstly, HE is a secure cryptographic encryption scheme, so using it to encrypt data during transmission is sufficient to thwart eavesdropping adversaries. Secondly, HE allows computation to be performed on the data without decryption, so it can prevent the exploitation of plain data. 
\begin{figure}[h!]
    \centering
    \includegraphics[width=0.8\textwidth]{figures/threatModel}
    \caption[The Threat Model]{A graphical representation of the threat model. Eve is an eavesdropper able to listen to communications between user and service provider. Mallory is a malicious actor within the service provider able to access users' video data. HE is able to obstruct both Eve and Mallory, preventing them from discovering the contents of the user's video.}
    \label{fig:threatModel}
\end{figure}






\section{Project Contributions}
\label{sec:projectContributions}
\indent \indent
This dissertation documents the design and implementation of a novel investigation into HE for video inference. It provides preliminary insight into the most challenging aspects of integrating the distinct fields of cryptography and computer vision to encourage further research. 
\begin{itemize}
    \item \textbf{Networking:} a client-server application simulating the device-server stack utilised by surveillance companies was constructed to enable the exploration of optimisations to increase the network throughput of videos encrypted using the CKKS HE scheme \cite{CKKS} provided by Microsoft's Secure Encrypted Arithmetic Library (SEAL) \cite{SEAL}.
    \item \textbf{Inference:} multiple inference algorithms were implemented to permit private and plain moving object detection, including investigating online GMMs following Stauffer and Grimson \cite{Stauffer} and the Expectation-Maximisation algorithm \cite{Dempster}.
    \item \textbf{MeKKS:} a HE implementation from first principles following the Homomorphic Encryption for Arithmetic of Approximate Numbers paper by Cheon et al.\ \cite{CKKS, BootstrappingHEAAN} to examine the benefits of specialising the implementation for video data by removing unneeded functionality, simplifying data structures, and vectorising ciphertexts.
\end{itemize}
 
\chapter{Preparation}
\label{chap:preparation}
\indent \indent
This chapter discusses the preparatory work done before beginning the project's implementation. §\ref{sec:preliminaries} introduces the relevant concepts, terminologies, and notations from cryptography, graphics, and unsupervised machine learning fundamental to the project, §\ref{sec:projectStrategy} discusses the methodologies used when approaching the design of the project's implementation, and §\ref{sec:startingPoint} provides an overview of the project's foundations.
\section{Preliminaries}
\label{sec:preliminaries}
\subsection{Homomorphic Encryption}
\label{sec:homomorphicEncryption}
\subsubsection{Introduction}
\indent \indent
There are two broad categories of cryptographic encryption schemes: private-key (symmetric) and public-key (asymmetric). While both can be applied to HE, this dissertation will address public-key encryption because it is the technique adopted by all HE schemes used in the project.
\smallskip \\ \indent
A public-key encryption scheme is defined by a triple of functions $\Pi = (\texttt{KeyGen}, \texttt{Enc}, \texttt{Dec})$. \texttt{KeyGen} is a function used to generate a \textit{public key} (\texttt{PK}) and \textit{private key}\footnote{The \textit{private key} is referred to as a \textit{secret key} by some literature. While these terms are equivalent, general convention is to use \textit{secret key} in relation to symmetric encryption, and \textit{private key} when discussing asymmetric.} (\texttt{SK}) such that $(\texttt{PK}, \texttt{SK}) \leftarrow \texttt{KeyGen}(1^l)$, where the security parameter, $l$, measures how hard it is for an adversary to break the scheme\footnote{An $l$-bit security parameter would require an expected $2^l$ attempts to guess the keys.}. Denoting the space of all possible plaintext messages as $\mathcal{M}$ and ciphertext messages as $\mathcal{C}$, a message $m \in \mathcal{M}$ is encrypted into its corresponding ciphertext $c \in \mathcal{C}$ by $c \leftarrow \texttt{Enc}_\texttt{PK}(m)$. Similarly $c$ is decrypted back into $m$ by $m \leftarrow \texttt{Dec}_\texttt{SK}(c)$.
\smallskip \\ \indent
In order to extend $\Pi$ into a HE scheme, a fourth function $\texttt{Eval}(f, c_1, \ldots, c_n)$ must be introduced. The evaluation function, $\texttt{Eval}$ applies a Boolean circuit, $f$, to the ciphertext arguments, $c_1, \ldots, c_n$ such that, for all arguments, it holds that
\begin{equation}
    \texttt{Dec}_\texttt{SK}(\texttt{Eval}(f, c_1, \ldots, c_n)) = f(m_1, \ldots, m_n)
\end{equation}
where $m_1, \ldots, m_n$ are the plaintext equivalents of $c_1, \ldots, c_n$. This is, perhaps, better illustrated by Figure \ref{fig:homomorphicEncryption} below.
\begin{figure}[ht]
    \centering
    \includegraphics[width=0.5\textwidth]{figures/homomorphicEncryption.png}
    \caption[Homomorphic Encryption]{Homomorphic Encryption}
    \label{fig:homomorphicEncryption}
\end{figure}
Theoretically, a \textit{fully} homomorphic scheme\footnote{The prefix \textit{fully} derives from the existence of \textit{partially} homomorphic schemes. A partially homomorphic scheme will only allow certain operations on the ciphertext - usually multiplication and division - and have existed for many years. Some examples of partially homomorphic schemes include RSA, ElGamal, and Paillier encryption ~\cite{PartialSchemes}.} allows the evaluation of Boolean circuits indefinitely. However, in practice, the time complexity of operations means many schemes are \textit{levelled} homomorphic schemes. This means they only support operations up to a \textit{bounded depth} - that is, only a predefined number of circuits can be applied to a ciphertext before the plaintext becomes irrecoverable. The maximum depth is a critical factor when applying HE to practical problems because it significantly limits the scope of supported algorithms.
\subsubsection{Ring Learning with Errors}
\indent \indent
For an encryption scheme to be \textit{perfectly secure}, a ciphertext must provide no additional information about its plaintext (see Equation \ref{eq:def1}). In other words, the probability of generating a given ciphertext from a particular plaintext is independent of the plaintext (see Equation \ref{eq:def2}). The two equations below can be shown to be equivalent using Bayes' rule. 
\begin{equation}
    \label{eq:def1}
    \probP(M = m \: | \: C = c) = \probP(C = c)
\end{equation}
\begin{equation}
    \label{eq:def2}
    \probP(C = c \: | \: M = m) = \probP(M = m)
\end{equation}
for all $m \in \mathcal{M}$, $c \in \mathcal{C}$.
\smallskip \\ \indent
While it is possible to create perfectly secure encryption schemes\footnote{For example, the One-Time Pad[OTP]}, they are impractical in real applications. Therefore, \textit{computational security} is considered sufficient. Relying on the hardness of certain mathematical problems, computational security means that an encryption scheme is \textit{practically unbreakable}: the most efficient known algorithm for breaking a cipher would require more computational steps than an attacker with unlimited hardware could perform. For example, the RSA encryption scheme relies on the fact that there exists no known, efficient algorithm for computing the prime factors of a large number on a classical computer - the integer factorisation problem. In complexity theory, this problem falls into the set of $\mathcal{NP}$.
\smallskip \\ \indent
Similarly, the HE schemes used in this project rely upon computational security. They utilise the hardness of the \textit{ring learning with errors} (henceforth RLWE) problem introduced by Lyubashevsky et al.\ ~\cite{RLWE}. A polynomial-time reduction from the \textit{shortest vector problem} to RLWE can be derived. Therefore, since the shortest vector problem is $\mathcal{NP}$-hard under the correct choice of parameters, it is safe to rely upon RLWE for computational security.
\smallskip \\ \indent
To understand the RLWE problem, knowledge of group theory is required. The relevant definitions are introduced in Appendix \ref{app:groups}. The RLWE problem requires the ring formed by the set of polynomials modulo $\Phi (X)$ that also have coefficients in $\mathbb{Z}_q$\footnote{$\mathbb{Z}_q$ is the set of integers modulo $q$. For example, $\mathbb{Z}_7 = \{0, 1, 2, 3, 4, 5, 6\}$.}. Known as a \textit{quotient ring}, this can be denoted by $\mathcal{R}_q = \mathbb{Z}[X] / (\Phi(X))$, where $\Phi(X)$ is an \textit{irreducible polynomial} - a polynomial which cannot be factored into two non-constant polynomials.
\smallskip \\ \indent
Informally, RLWE describes the problem of finding an unknown $s \in \mathcal{R}_q$ given a polynomial vector computed using $s$ and some sampled errors. An encryption scheme can be created such that, after encoding a plaintext vector, $\vec{v}$, as a list of polynomials and it to a polynomial using a secret polynomial, it is infeasible to recover $\vec{v}$ in polynomial time.
\smallskip \\ \indent
The HE schemes discussed in this dissertation rely upon the RLWE problem to assert \textit{indistinguishable encryptions under a chosen-plaintext attack} (IND-CPA) security. Fundamentally, any encryption scheme is IND-CPA secure if, when attacked by a probabilistic, polynomial-time adversary, the chances of correctly deciding which of two plaintexts a particular ciphertext corresponds to is even. Therefore, in practice, even if an adversary has access to an encryption \textit{oracle}, the adversary's chances of calculating the correct plaintext when given a ciphertext should be no better than if they were randomly guessing. 
\smallskip \\ \indent
Practically, HE schemes choose a \textit{cyclotonic polynomial} for $\Phi(X)$, where the $n$-th cyclotonic polynomial, $\Phi_n(X)$, is defined as
\begin{equation}
    \Phi_n(X) = \prod_{k \in [1, n]; \; gcd(k, n = 1)} X - e^\frac{2 i \pi k}{n}
\end{equation}
In order to speed up computation, $n$ is selected to be an even power of two. Consequently, $\Phi_n(X) = X^\frac{n}{2} + 1$. This allows the \textit{number theoretic transform} (NTT)\footnote{A specialisation of the discrete Fourier transform, the NTT is a generalisation of the Fast Fourier transform in the case of finite fields. ~\cite{NTT}}. The advantage of this is that it can be easily accelerated using hardware ~\cite{Hardware}.
\smallskip \\ \indent
The sampled errors used when deriving ciphertexts imply almost exponential error growth in the number of multiplications applied. The limited depth property of levelled HE schemes directly results from this. However, the relative growth size can be reduced by increasing the modulus $q$. Although, if a polynomial degree of size $n/2$ is used, efficient attacks exist against the RLWE problem for a small value of $q$ ~\cite{HEStandard}. Therefore, a fundamental trade-off is introduced between the supported depth of multiplication and the security level.





\subsection{Moving Object Detection}
\label{sec:movingObjectDetection}
\subsubsection{Introduction}
\indent \indent
\textit{Image segmentation} is a well-established problem in the fields of digital image processing and computer vision research. The problem describes the process of partitioning a digital image into multiple regions, represented as sets of pixels. The goal is to simplify an image representation so that it is easier to, for example, locate objects or boundaries, allowing further analysis. More precisely, image segmentation describes labelling each pixel of an image such that all pixels sharing a particular characteristic are assigned the same label ~\cite{Shapiro}.
\smallskip \\ \indent
There are two broad categories of segmentation techniques: \textit{semantic} \cite{Semantic} and \textit{instance} \cite{Instance}. This dissertation will focus on semantic segmentation through moving object detection because videos will be segmented into two categories: foreground and background, rather than labelling individual instances of objects.
\smallskip \\ \indent
To perform \textit{foreground extraction}, also known as \textit{background subtraction}, the background of an image must be modelled so that changes in the scene can be detected. This is an unsolved problem. Image data can be very diverse, with variable lighting and repetitive movements (like leaves and shadows) making robust models hard to develop.
\smallskip \\ \indent
Once a background model has been developed, it can be used to detect moving objects in videos by comparing each frame to a reference frame and extracting the differences. There are several methods for achieving this. Below are five algorithms investigated for this dissertation.
\subsubsection{Frame Differencing}
\indent \indent
\textit{Frame differencing} is the most straightforward background subtraction algorithm ~\cite{FrameDifferencing}. First, a reference frame is established. There are several options for this, including selecting the first frame, or continually updating to the frame previously received. Challenges include ensuring the model can adapt to permanent changes, like a fence being added to a property, and accounting for random noise. Updating the reference too frequently can cause issues with slow-moving objects becoming occluded.
\smallskip \\ \indent
Once the reference frame, $B$, has been selected, the foreground can be extracted. Denoting each frame at time $t$ as $f_t$, the value of each pixel in $B$, $P(B)$, can be subtracted from the corresponding pixel in $f_t$, $P(f_t)$. The mathematical definition is given by Equation\ref{eq:frameDifferencing}.
\begin{equation}
    \label{eq:frameDifferencing}
    P(F_t) = P(f_t) - P(B)
\end{equation}
where $F$ represents the frames in the resultant video.
\subsubsection{Mean Filter}
\indent \indent
A \textit{mean filter} approach to moving object detection attempts to overcome the weaknesses of frame differencing in selecting a reference frame ~\cite{MeanFilter}. Instead of taking a frame directly from the video, the value of $B$ at time $t$ is calculated using Equation \ref{eq:meanFilter}.
\begin{equation}
    \label{eq:meanFilter}
    B = \frac{1}{N} \sum^N_{i=1} f_{t-i}
\end{equation}
where $N$ is the number of preceding images included in the average, and $f_t$ is the frame in the video at time $t$. $N$ would depend on the video speed and the amount of motion expected in the video.
\smallskip \\ \indent
After $B$ has been calculated at time $t$, the value of the resultant video $F_t$ can be calculated using the same method as frame differencing, given by Equation \ref{eq:frameDifferencing}.
\subsubsection{Median Filter}
\indent \indent
The \textit{median filter} algorithm is similar to performing mean filter extraction. Instead of using the mean to calculate the reference frame, the median of the preceding $N$ frames is used ~\cite{MeanFilter}. Then, the moving objects are extracted by subtracting the reference frame from each frame in the video, according to Equation \ref{eq:frameDifferencing}.
\subsubsection{Gaussian Average}
\label{sec:gaussianAverage}
\indent \indent
Wren et al.\ originally proposed fitting a Gaussian probabilistic density function to the most recent $N$ frames ~\cite{Wren}. Rather than storing an image as the reference frame, this method stores a mean and variance for each pixel. When a new frame is received, the likelihood of each pixel is calculated. Assuming that the background is the most common value for a pixel, the foreground can be segmented by grouping the pixels with a sufficiently low likelihood.
\smallskip \\ \indent
Rather than recalculating the mean every time a frame is received - giving an $O(n^2)$ time complexity - the algorithm can be implemented using cumulative functions for the mean and variance to achieve a linear time complexity. The functions are defined by Equation \ref{eq:mean} and Equation \ref{eq:variance} respectively. 
\begin{equation}
    \label{eq:mean}
    \mu_t =
    \begin{cases}
        f_0 & \text{if $t = 0$} \\
        \alpha f_t + (1 - \alpha) \mu_{t-1} & \text{otherwise}
    \end{cases}
\end{equation}
\begin{equation}
    \label{eq:variance}
    \sigma^2_t =
    \begin{cases}
        c & \text{if $t = 0$} \\
        d^2 \alpha + (1 - \alpha) \sigma^2_{t-1} & \text{otherwise}
    \end{cases}
\end{equation}
where $\alpha$ determines the size of the \textit{temporal window} used to fit the Gaussian model\footnote{$\alpha$ weights frames according to age. Eventually, an old frame will be weighted so insignificantly that its impact is negligible. Therefore, it determines how far into the past the model uses to predict future pixels.}, $d = |f_t - \mu_t|$ gives the Euclidean distance from the pixel to the mean, and $c$ is some constant defined by the model creator. 
\smallskip \\ \indent
From these models, the foreground can be extracted according to Equation \ref{eq:meanThreshold},
\begin{equation}
    \label{eq:meanThreshold}
    \frac{|f_t - \mu_t|}{\sigma_t} > k
\end{equation}
where $k$ is a constant that the model creator can tune to achieve optimal results.
\smallskip \\ \indent
A variant of this algorithm might only update the model if a pixel is believed to be in the background. This prevents the model from becoming skewed if there is lots of movement. However, it requires the entire frame to be initially background and struggles to cope with constant changes.
\subsubsection{Gaussian Mixture Models}
\indent \indent
Stauffer and Grimson proposed \textit{Gaussian mixture models} (henceforth GMMs) for moving object detection in 1999 ~\cite{Stauffer}. GMMs are probabilistic models that represent the presence of normally distributed subpopulations within an overall population. They are particularly useful because they don't require the subpopulation of a data point to be identified. Instead, subpopulations are labelled automatically, constituting \textit{unsupervised machine learning}.
\smallskip \\ \indent
There are two types of parameters necessary for GMMs, the \textit{component weights}, and the component \textit{means} and \textit{variances}. For a GMM with $N$ components, the $i^\text{th}$ component has $\mu_i$ and variance $\sigma_i$ in the \textit{univariate case}, and mean $\vec{\mu}_i$ and a covariance matrix $\Sigma_i$ in the \textit{multivariate case}. The component weights, $\phi_k$ for component $k$, are constrained by the equation $\sum^K_{i=1} \phi_i = 1$. If the component weights aren't learned, they are known as an \textit{a-priori}\footnote{A probability derived purely through deductive reasoning.} distribution over components such that $\probP(x \text{ generated by component k}) = \phi_k$. If the component weights are learned, they are known as \textit{a-posteriori}\footnote{From Bayesian statistics, referring to conditional probability $\probP(A | B)$.} estimates of the component probabilities given the data.
\smallskip \\ \indent
When fitting a GMM, the Gaussian distributions are tuned to match the distributions observed in the data. A common method for this is to use \textit{expectation maximisation} (EM), if the number of components is known. If all of the data is available, it can be incorporated into this stage to achieve the most accurate fitting. However, only a subset of the data will likely be available in practice, so the GMM will extrapolate to more values. This technique can also be taken advantage of if there is limited time to perform fitting.
\smallskip \\ \indent
Once a GMM has been fitted, it can be used for inference. This dissertation will focus on inference for \textit{clustering} because it is most useful for moving object detection. The posterior component assignment probabilities can be estimated using Bayes' theorem over the GMM parameters. Knowing the component that a data point most likely belongs to provides a way to group the points into clusters. In the scenario of moving object detection, this would be two clusters: the foreground and the background.






\section{Project Strategy}
\label{sec:projectStrategy}
\subsection{Requirements Analysis}
\label{sec:requirements}
\indent \indent
The requirements for this project are listed below. The project required the development of theoretical knowledge before implementation began so the requirements evolved as understanding matured. The original requirements are given in Appendix \ref{app:proposal} for comparison. The requirements have been grouped into two categories. The first, labelled $A$, are the core requirements essential to the project's success. The second, labelled $B$, are extensions, aiming to improve understanding or further the investigation into HE and surveillance.
\begin{table}[h!]
    \centering
    % \resizebox{\textwidth}{!}{%
    \begin{tabular}{|>{\centering\arraybackslash}m{0.5cm}||m{10.9cm}|>{\centering\arraybackslash}m{1.63cm}|>{\centering\arraybackslash}m{1.91cm}|}
        \hline
        & \textrm{\textbf{Requirement}} & \textrm{\textbf{Priority}} & \textrm{\textbf{Difficulty}} \\
        \hline \hline
        \texttt{A1} & \textrm{Implement a client-server application allowing videos to be homomorphically encrypted and transmitted in both directions.} & \color{Red}\textrm{High} & \color{Dandelion}\textrm{Medium} \\
        \hline
        \texttt{A2} & \textrm{Implement background subtraction models that can extract moving objects from homomorphically encrypted videos.} & \color{Red}\textrm{High} & \color{Dandelion}\textrm{Medium} \\
        \hline
        \texttt{A3} & \textrm{Evaluate the accuracy of HE inference to investigate its applicability to real systems.} & \color{Red}\textrm{High} & \color{Green}\textrm{Low} \\
        \hline \hline
        \texttt{B1} & \textrm{Implement a bespoke HE scheme and integrate it into the core application, providing the same functionality as the established scheme already used.} & \color{Dandelion}\textrm{Medium} & \color{Red}\textrm{High} \\
        \hline
        \texttt{B2} & \textrm{Analyse the security of the encryption schemes used in the project.} & \color{Green}\textrm{Low} & \color{Red}\textrm{High} \\
        \hline
        \texttt{B3} & \textrm{Implement a deep-learning object recognition algorithm acting on HE data following Cryptonets \cite{Dowlin}.} & \color{Green}\textrm{Low} & \color{Red}\textrm{High} \\
        \hline
    \end{tabular}%}
    \caption[Requirements Analysis]{Requirements analysis.}
\end{table}
\subsection{Methodology}
\indent \indent
Different stages of the project were best suited to different development methodologies. For core components, a waterfall methodology was adopted ~\cite{Waterfall}. The requirements were detailed and unambiguous, so the project lent itself to a structured methodology, not requiring the flexibility of an iterative approach. The model's stages are detailed below.
\smallskip \\ \indent
The results of the \textit{requirements analysis} stage have been provided in §\ref{sec:requirements}. This stage is where most of the research was performed to ensure the project's design was better informed.
\smallskip \\ \indent
The \textit{design} phase incorporates expanding on the requirements into a physical project; this includes, for example, creating the class diagrams shown in Figure \ref{fig:clientUML}, Figure \ref{fig:serverUML}, and Figure \ref{fig:mekksUML}.
\smallskip \\ \indent
The \textit{implementation} and \textit{testing} stages were intertwined where possible in order to promote a test-driven approach to development. This was made easier by the object-oriented methodology and unit testing practices adopted.
\smallskip \\ \indent
The \textit{evaluation} stage replaces the \textit{maintenance} stage of the traditional Waterfall model. This stage involves running experiments to evaluate the project. 
\smallskip \\ \indent
When working on extensions, an iterative model was more appropriate. The main reason for this was that less time had been dedicated to researching these components, so implementation was riskier. Consequently, a rapid cyclical development model requiring components to be decomposed would allow any problems to be discovered sooner, limiting impact. Therefore, the Agile model, depicted in Figure \ref{fig:agile} was selected. Regular supervisor meetings allowed Agile's sprint system to be utilised so that a project prototype could be presented in each meeting to ensure thorough progression tracking.
\begin{figure}[h!]
    \centering
    \begin{subfigure}[b]{0.495\textwidth}
        \centering
        \includegraphics[width=1.1\textwidth]{figures/waterfall.png}
        \caption{The Waterfall Development Model}
        \label{fig:waterfall}
    \end{subfigure}
    \hfill
    \begin{subfigure}[b]{0.495\textwidth}
        \centering
        \includegraphics[width=0.9\textwidth]{figures/agile.png}
        \caption{The Agile Development Model}
        \label{fig:agile}
    \end{subfigure}
    \caption{Development Models}
\end{figure}
\begin{figure}[h!]
    \centering
    \includegraphics[width=\textwidth]{figures/gantt.png}
    \caption{Project Timeline}
    \label{fig:gantt}
\end{figure}
\smallskip \\ \indent
A Gantt chart of the project's timeline is shown in Figure \ref{fig:gantt}.
\subsection{Testing}
\label{sec:testing}
\indent \indent
Unlike traditional software engineering, machine learning does not provide precise criteria against which correctness can be verified. The models used for background subtraction are probabilistic, so the outputs cannot be precisely predicted.  Consequently, a variety of testing methodologies were required.
\subsubsection{Unit Tests}
\indent \indent
Designed for testing atomic units of source code, unit testing utilises the independence resulting from the object-oriented design approach to test components in isolation. Test cases provide expected, boundary, and erroneous data to ensure function results match the expected. Unit tests can be automated to allow repeated checks as changes to the source code are made, ensuring errors aren't introduced.
\smallskip \\ \indent
Unit tests were particularly useful when completing the first extension for verifying the correctness of the encoding, encryption, and decryption functions and the HE Boolean circuits.
\subsubsection{Integration Tests}
\indent \indent
Integration tests increase the scope of functionality covered by each test by ensuring separate modules interact correctly. Once unit testing has been completed, these tests aggregate verified modules and provide data to ensure the output is correct.
\smallskip \\ \indent
Integration testing was useful in verifying that the software stack functioned correctly. For example, ensuring the client and server communicated correctly. While some integration testing can be automated, more complex engineering work was prioritised over creating a comprehensive testing suite, so manual integration testing was primarily used.
\subsubsection{Manual Verification}
\indent \indent
Manual verification was used to overcome the challenges of testing the background subtraction models. Since the project involves video data, human inspection provides a good intuition of whether or not a background has been correctly removed. If a more detailed analysis is required, pixel values can be compared to check for expected results or verify consistency across multiple tests.






\section{Starting Point}
\label{sec:startingPoint}
\subsection{Knowledge and Experience}
\indent \indent
Prior to beginning the project, the following relevant Tripos courses had been completed: \textit{Scientific Computing}, \textit{Machine Learning and Real-world Data}, \textit{Software and Security Engineering}, \textit{Concurrent and Distributed Systems}, \textit{Data Science}, \textit{Computer Networking}, and \textit{Security}. The Part II course, \textit{Cryptography} was also useful in understanding the theoretical underpinnings of encryption.
\smallskip \\ \indent
However, it should be noted that HE is not included in the scope of the Cryptography course, so theory was learned independently of Tripos studies. The study of applied HE is sparsely documented, so most understanding came from academic papers; notably \cite{CKKS} and \cite{SEAL}. Although, articles such as \cite{BrilliantHE} were more useful for foundational knowledge.
\smallskip \\ \indent
Similarly, there was little mention of computer vision artificial intelligence in Tripos, so most understanding came from independent research. Academic papers such as \cite{Stauffer} and \cite{Kulchandani} were helpful, particularly when considering privacy-preserving computer vision. Some understanding also came during a summer internship completed in the field of object recognition deep learning.
\subsection{Tools Used}
\subsubsection{Programming Languages}
\indent \indent
All code written for this dissertation was written in \textit{Python} ~\cite{Python}. The main reasons for this were the large machine learning ecosystem, ease of use, and ease of debugging. These factors allowed for quick implementation, making Python best suited to the project's tight schedule. 
\smallskip \\ \indent
However, Python is not a language traditionally used for cryptographic applications. Usually, lower-level, faster languages like C++ are favoured. Since the focus of the project was investigating the efficacy of moving object detection in the HE domain, the speed of execution was not prioritised over the speed of implementation.
\subsubsection{Software Development}
\indent \indent
\textit{Visual Studio Code} \cite{VSCode} development environment was used for writing code because of support for Python as well as a wide variety of plugins that allow integration of other valuable tools such as ESLint.  In addition, \textit{Git} \cite{Git} and \textit{GitHub} \cite{Github} were used for version control and source code management. \textit{OneDrive} \cite{OneDrive} was also used to hold another backup for safety.
\subsubsection{Encryption Schemes}
\indent \indent
The project focuses on HE schemes based on the RLWE problem. The main reason for this was the availability of academic literature discussing them. In particular, the CKK scheme \cite{CKKS} was selected because it supports representing real numbers\footnote{Versus the BFV scheme, which only supports integers ~\cite{BFV1, BFV2}.}. However, the project is designed to allow any HE scheme following the same API to be substituted in CKKS's place.
\subsubsection{Libraries}
\indent \indent
The project uses Microsoft's SEAL library \cite{SEAL}, which provides a C++ implementation of the CKKS scheme. This was chosen because of the extensive optimisations that have been applied. In particular, SEAL uses a residue-number-system variant of CKKS to support large plaintext moduli. SEAL was integrated using a Python wrapper library ~\cite{Wrapper}.
\subsubsection{Datasets}
\indent \indent
There were two publicly available datasets used in this project:
\begin{itemize}
    \item The Moving-MNIST dataset contains ten thousand sequences of length twenty frames showing two handwritten digits from the standard MNIST dataset moving in a $64 \times 64$ pixel frame ~\cite{MovingMNIST, MNIST}. This is a relatively simple dataset to perform moving object detection on because it only contains white objects on a black background. Therefore, it was useful in providing a baseline for the performance of the inference algorithms.
    \item The LASIESTA dataset contains sequences showing individuals moving across static backgrounds ~\cite{LASIESTA}. Specifically designed to evaluate segmentation algorithms, this dataset provides more realistic examples of surveillance footage so allows a truer evaluation of moving object detection in the HE domain.
\end{itemize}
\subsubsection{Licensing}
\indent \indent
All software dependencies in this project use permissive libraries that allow their code to be used without restrictions. The same is true for the datasets. Table \ref{tab:licensing} gives the specific licenses.
\begin{table}[h!]
\centering
\resizebox{0.5\textwidth}{!}{%
\begin{tabular}{|p{4cm}|p{4cm}|}
    \hline
    \textrm{\textbf{Dependency}} & \textrm{\textbf{Licence}} \\
    \hline \hline
    \texttt{Multiprocessing} & \multirow{3}{*}{\textrm{3-Clause BSD ~\cite{BSD}}} \\ 
    \texttt{NumPy} & \\ 
    \texttt{SymPy} & \\
    \hline
    \texttt{Microsoft SEAL} & \multirow{3}{*}{\textrm{MIT ~\cite{MIT}}} \\
    \texttt{SEAL-Python} & \\
    \texttt{PyJoules} & \\
    \hline
    \texttt{Matplotlib} & \multirow{3}{*}{\textrm{PSFL ~\cite{PSFL}}} \\ 
    \texttt{Python} & \\
    \texttt{Tkinter} & \\
    \hline
\end{tabular}}
\caption{Licenses}
\label{tab:licensing}
\end{table}
\subsection{Computer Resources}
\indent \indent
The original project proposal mentioned that external computational resources might have been required during the implementation phase, such as AWS or Microsoft Azure. However, the project was entirely developed, tested, and evaluated on a MacBook Pro laptop. The specifications are listed below, in Table \ref{tab:specs}.
\begin{table}[h!]
\centering
\begin{tabular}{>{\hspace{1em}}l l}
    \myheading{\textrm{Processor}}
    \textrm{CPU}                       & \SI{8}{\; Cores}                                            \\
    \textrm{GPU}                       & \SI{14}{\; Cores}                                           \\
    \textrm{Neural Engine}             & \SI{16}{\; Cores}                                           \\
    \textrm{Memory Bandwidth}          & \SI{200}{\; GB \per s}                                      \\
    \myheading{\textrm{Memory}}
    \textrm{RAM}                       & \SI{32}{\; GB \text{ (unified memory)}}                     \\
\end{tabular}
\caption{Computer Specifications}
\label{tab:specs}
\end{table}

\chapter{Implementation}
\label{chap:implementation}

What you actually did

\begin{figure}[ht]
    \centering
    \input{classes}
    \caption{UML Class Diagram showing the components of the software.}
    \label{fig:classDiagram}
\end{figure}

\section{Implementation}


\section{Evaluation}

\chapter{Evaluation}
\label{chap:evaluation}

\indent \indent
This chapter evaluates the project's implementation using three main criteria: the extent to which it meets the success criteria detailed in §\ref{sec:requirements},  the applicability of HE to inference algorithms, and its practicality regarding current, real-world surveillance technology. Consequently, the chapter is divided into three sections to tackle each criterion distinctly. Both quantitative and qualitative analysis is used throughout the chapter to analyse the implementation's performance and make predictions about the scope to which the investigation may be extended in future. Unless otherwise specified, the data presented was generated using $32 \times 32$ pixel images from the Moving-MNIST dataset.

\section{Requirements Analysis}
\begin{table}
    \centering
    \def\arraystretch{1.25}
    \begin{tabular}{|c|c|p{9.6cm}|}
        \hline
        \textrm{\textbf{Requirement}} & \textrm{\textbf{Achieved?}} & \textrm{\textbf{Justification}} \\
        \hline \hline
        \texttt{A1} & \cmark & \textrm{The project contains a client-server application that allows videos to be homomorphically encrypted and transmitted across a network, with implementation techniques detailed in §\ref{sec:networking}.} \\
        \hline
        \texttt{A2} & \cmark & \textrm{The project contains several algorithms that are able to extract moving objects from homomorphically encrypted videos, with implementation techniques detailed in §\ref{sec:inference}.} \\
        \hline
        \texttt{A3} & \cmark & \textrm{The accuracy of HE inference algorithms are evaluated to investigate their efficacy and applicability in §\ref{sec:integration}.} \\
        \hline \hline
        \texttt{B1} & \cmark & \textrm{The MeKKS scheme, detailed in §\ref{sec:mekks}, provides a complete implementation of the fundamental principles of the CKKS HE scheme.} \\
        \hline
        \texttt{B2} & \xmark &  \textrm{Due to time constraints, an independent investigation into the security of HE schemes could not be completed. However, only well-established, trusted schemes were considered; hence CKKS was selected and reimplemented over less secure schemes.} \\
        \hline
        \texttt{B3} & \xmark & \textrm{The implementation of moving object detection algorithms proved to be more open than expected. Consequently, more time was dedicated to further understanding this area rather than expanding into other inference paradigms.} \\
        \hline
    \end{tabular}
    \caption[Requirements Analysis]{Requirements analysis.}
    \label{tab:requirements}
\end{table}
\setlength{\leftskip}{0.25cm}
\indent \indent
The success criteria in §\ref{sec:requirements} were split into two categories: \textit{core} and \textit{extensions}.  As detailed in Table \ref{tab:requirements}, All three of the core criteria were implemented, and one of the three extensions has also been completed. The open-ended nature of this project means that defining a \textit{completed} state for some criterium was not trivial. For example, for criterium \texttt{A2}, while some algorithms have been implemented in their entirety, others require further investigation. However, it was important to have a goal for each criterium to properly plan the project and consider all aspects equally. Therefore, a justification for the state of each criterium has been included.

\setlength{\leftskip}{0cm}




\section{Homomorphic Encryption Integration}
\label{sec:integration}
\setlength{\leftskip}{0.25cm}
\indent \indent
Adapting moving object detection algorithms for the HE domain proved to be the project's biggest challenge. The limited number of operations available, combined with the limited number of applications supported by ciphertexts, means some aspects of inference cannot be recreated. Particular challenges came when trying to implement the median filter and a GMM.
\smallskip \\ \indent
However, frame differencing, the mean filter, and the Gaussian average methods of background subtraction were successfully implemented using the techniques described in §\ref{sec:adaptations}. A sample of the results from each of these algorithms running on the Moving-MNIST dataset is provided in Figure \ref{fig:mnistInferenceResults}, and an example of what can be achieved on a more realistic dataset - the LASIESTA dataset - is provided in Figure \ref{fig:lasiestaInferenceResults}.
\begin{figure}
    \centering
    \input{figures/mnistInferenceResults}
    \caption{Moving-MNIST Inference Results}
    \label{fig:mnistInferenceResults}
\end{figure}
\begin{figure}
    \centering
    \begin{subfigure}[t]{0.9\textwidth}
        \centering
        \includegraphics[scale=0.7]{figures/lasiesta/frame0}
        \hfill
        \includegraphics[scale=0.7]{figures/lasiesta/frame100}
        \hfill
        \includegraphics[scale=0.7]{figures/lasiesta/frame190}
        \hfill
        \includegraphics[scale=0.7]{figures/lasiesta/frame250}
        \hfill
        \includegraphics[scale=0.7]{figures/lasiesta/frame270}
        \caption{Original LASIESTA frames.}
    \end{subfigure}
    \\ \bigskip
    \begin{subfigure}[t]{0.9\textwidth}
        \centering
        \includegraphics[scale=0.7]{figures/LASIESTA-PLAIN-DIFFERENCING/frame0}
        \hfill
        \includegraphics[scale=0.7]{figures/LASIESTA-PLAIN-DIFFERENCING/frame100}
        \hfill
        \includegraphics[scale=0.7]{figures/LASIESTA-PLAIN-DIFFERENCING/frame190}
        \hfill
        \includegraphics[scale=0.7]{figures/LASIESTA-PLAIN-DIFFERENCING/frame250}
        \hfill
        \includegraphics[scale=0.7]{figures/LASIESTA-PLAIN-DIFFERENCING/frame270}
        \caption{Frame Differencing with no HE.}
    \end{subfigure}
    \\ \bigskip
    \begin{subfigure}[t]{0.9\textwidth}
        \centering
        \includegraphics[scale=0.7]{figures/LASIESTA-PLAIN-MEAN/frame0}
        \hfill
        \includegraphics[scale=0.7]{figures/LASIESTA-PLAIN-MEAN/frame100}
        \hfill
        \includegraphics[scale=0.7]{figures/LASIESTA-PLAIN-MEAN/frame190}
        \hfill
        \includegraphics[scale=0.7]{figures/LASIESTA-PLAIN-MEAN/frame250}
        \hfill
        \includegraphics[scale=0.7]{figures/LASIESTA-PLAIN-MEAN/frame270}
        \caption{Mean Filter with no HE.}
    \end{subfigure}
    \\ \bigskip
    \begin{subfigure}[t]{0.9\textwidth}
        \centering
        \includegraphics[scale=0.7]{figures/LASIESTA-PLAIN-GAUSSIAN/frame0}
        \hfill
        \includegraphics[scale=0.7]{figures/LASIESTA-PLAIN-GAUSSIAN/frame100}
        \hfill
        \includegraphics[scale=0.7]{figures/LASIESTA-PLAIN-GAUSSIAN/frame190}
        \hfill
        \includegraphics[scale=0.7]{figures/LASIESTA-PLAIN-GAUSSIAN/frame250}
        \hfill
        \includegraphics[scale=0.7]{figures/LASIESTA-PLAIN-GAUSSIAN/frame270}
        \caption{Gaussian Average with no HE.}
    \end{subfigure}
    \\ \bigskip
    \begin{subfigure}[t]{0.9\textwidth}
        \centering
        \includegraphics[scale=0.7]{figures/LASIESTA-CKKS-DIFFERENCING/frame0}
        \hfill
        \includegraphics[scale=0.7]{figures/LASIESTA-CKKS-DIFFERENCING/frame100}
        \hfill
        \includegraphics[scale=0.7]{figures/LASIESTA-CKKS-DIFFERENCING/frame190}
        \hfill
        \includegraphics[scale=0.7]{figures/LASIESTA-CKKS-DIFFERENCING/frame250}
        \hfill
        \includegraphics[scale=0.7]{figures/LASIESTA-CKKS-DIFFERENCING/frame270}
        \caption{Frame Differencing with CKKS.}
    \end{subfigure}
    \\ \bigskip
    \begin{subfigure}[t]{0.9\textwidth}
        \centering
        \includegraphics[scale=0.7]{figures/LASIESTA-CKKS-MEAN/frame0}
        \hfill
        \includegraphics[scale=0.7]{figures/LASIESTA-CKKS-MEAN/frame100}
        \hfill
        \includegraphics[scale=0.7]{figures/LASIESTA-CKKS-MEAN/frame190}
        \hfill
        \includegraphics[scale=0.7]{figures/LASIESTA-CKKS-MEAN/frame250}
        \hfill
        \includegraphics[scale=0.7]{figures/LASIESTA-CKKS-MEAN/frame270}
        \caption{Mean Filter with CKKS.}
    \end{subfigure}
    \\ \bigskip
    \begin{subfigure}[t]{0.9\textwidth}
        \centering
        \includegraphics[scale=0.7]{figures/LASIESTA-CKKS-GAUSSIAN/frame0}
        \hfill
        \includegraphics[scale=0.7]{figures/LASIESTA-CKKS-GAUSSIAN/frame100}
        \hfill
        \includegraphics[scale=0.7]{figures/LASIESTA-CKKS-GAUSSIAN/frame190}
        \hfill
        \includegraphics[scale=0.7]{figures/LASIESTA-CKKS-GAUSSIAN/frame250}
        \hfill
        \includegraphics[scale=0.7]{figures/LASIESTA-CKKS-GAUSSIAN/frame270}
        \caption{Gaussian Average with CKKS.}
    \end{subfigure}

    \caption{LASIESTA Inference Results}
    \label{fig:lasiestaInferenceResults}
\end{figure}
\setlength{\leftskip}{0cm}
\subsection{Online Mixture Model}
\setlength{\leftskip}{0.5cm}
\indent \indent
In the \textit{online mixture model} algorithm detailed in §\ref{sec:OMM}, Equation \ref{eq:gmmInequality} describes how a fitted model can be used to segment an image. However, inequality comparison operators are not provided by the standard CKKS implementation. To solve this, Cheon et al.\ proposed the algorithm in Figure \ref{fig:comparison} ~\cite{Comparison}. Unfortunately, this introduces security concerns. If the ability to compare two HE ciphertexts is added to the system, an attacker\footnote{such as Mallory in §\ref{sec:threatModel}.} could use it to exfiltrate information about the image. For example, with enough comparison operations, they would be able to determine the exact value of each pixel in a frame. Consequently, this algorithm was abandoned to preserve the security of the system.
\begin{figure}
    \centering
    \includegraphics[width=\textwidth]{figures/algorithm1}
    \caption{Homomorphic Comparison Algorithm}
    \label{fig:comparison}
\end{figure}
\smallskip \\ \indent
For the same reason, the median filter could not be implemented. The pixel values could not be ordered without a comparison operator, so a median could not be calculated.


\setlength{\leftskip}{0cm}
\subsection{Expectation-Maximisation Algorithm}
\setlength{\leftskip}{0.5cm}
\indent \indent
The difficulty in implementing this algorithm came from the number of operations that need to be performed across the fitting and predicting stages. Calculating means and covariances repeatedly requires many successive multiplications, requiring many coefficient levels in ciphertexts. Also, like the online mixture model, not all operations are supported by the CKKS scheme. In particular, the algorithm requires several divisions to be performed when updating the Gaussian distributions. Currently, the best solution for this seems to be provided by Cheon et al.\ ~\cite{Comparison}. However, the algorithm, given in Figure \ref{fig:division}, has a minimal domain requiring input values to be between zero and two. Normalising pixel values might provide a method for incorporating this, but the noise induced by HE means inference becomes infeasibly inaccurate
\begin{figure}
    \centering
    \includegraphics[width=\textwidth]{figures/algorithm2}
    \caption{Homomorphic Division Algorithm}
    \label{fig:division}
\end{figure}

\setlength{\leftskip}{0cm}




\section{Practicality}
\setlength{\leftskip}{0.25cm}
\indent \indent
This section will evaluate the implementation from the perspective of practicality in real-world surveillance systems. It will do so through two key aspects: the MLaaS client-server model and the accuracy of inference algorithms.

\setlength{\leftskip}{0cm}
\subsection{Networking}
\subsubsection{Data Handling}
\setlength{\leftskip}{0.5cm}
\indent \indent
Before data can be sent across the network, it must be \textit{packed}, and once it is received, it must be \textit{unpacked}. This proved to be a significant bottleneck before transmitting data. During the packing phase, data must be encrypted and serialised before it is transmitted over the network. To make transmission more efficient, a compression stage is added to try and reduce the memory usage of videos. Similarly, in the unpacking phase, data must be decompressed, deserialised, and decrypted to recover the video and inference results.
\smallskip \\ \indent
As described in §\ref{sec:networking}, several methods were investigated to try and reduce this bottleneck. By combining some of these techniques, substantial progress was made in reducing the time the packing and unpacking algorithms took to run. Figure \ref{fig:naivePackingAndUnpackingGraph} provides the running time of a naw\"ive implementation of these algorithms, and Figure \ref{fig:packingAndUnpackingGraph} demonstrates the performance of an optimised implementation. From these charts, Table \ref{tab:packingAndUnpacking} has been derived to highlight the improvement for each category of inference and encryption scheme. Interestingly, the unpacking algorithm can be improved using parallelisation when the CKKS scheme is used, but it will worsen performance when MeKKS is used. This is due to the delays caused by deserialising data that were overcome by implementing directly in Python - although this does make the encryption and decryption functions perform dramatically worse.
\smallskip \\ \indent
While these times may appear slow, it is important to remember that surveillance companies rarely stream all video from a device. Cameras will usually contain multiple sensors to determine which video is worth performing inference on to conserve battery life. Consequently, real-time performance is not required.

\begin{figure}
    \centering
    \begin{subfigure}[b]{0.495\textwidth}
        \centering
        \includegraphics[width=\textwidth]{figures/naivePackingTimes.eps}
        \caption{Packing}
    \end{subfigure}
    \hfill
    \begin{subfigure}[b]{0.495\textwidth}
        \centering
        \includegraphics[width=\textwidth]{figures/naiveUnpackingTimes.eps}
        \caption{Unpacking}
    \end{subfigure}
    \caption{Na\"ive Packing and Unpacking Times}
    \label{fig:naivePackingAndUnpackingGraph}
\end{figure}

\begin{figure}
    \centering
    \begin{subfigure}[b]{0.495\textwidth}
        \centering
        \includegraphics[width=\textwidth]{figures/packingTimes.eps}
        \caption{Packing}
    \end{subfigure}
    \hfill
    \begin{subfigure}[b]{0.495\textwidth}
        \centering
        \includegraphics[width=\textwidth]{figures/unpackingTimes.eps}
        \caption{Unpacking}
    \end{subfigure}
    \caption{Optimised Packing and Unpacking Times}
    \label{fig:packingAndUnpackingGraph}
\end{figure}

\begin{table}
    \centering
    \def\arraystretch{1.25}
    \resizebox{\textwidth}{!}{%
    \begin{tabular}{|c|c|c|c|c|c|}
        \hline
        \textrm{\textbf{Encryption}} & \textrm{\textbf{Inference}} & \textrm{\textbf{Process}} & \textrm{\textbf{Na\"ive} (s)} & \textrm{\textbf{Optimised} (s)} & \textrm{\textbf{Improvement}}
        \\ \hline \hline
        \multirow{6}{*}{\textrm{CKKS}}  & \multirow{2}{*}{\textrm{Differencing}} & \textrm{Packing}   & $6.440$ & $0.198$ & $32.5 \times$
        \\ \cline{3-6} 
                                        &                                        & \textrm{Unpacking} & $17.349$ & $0.850$ & $20.4 \times$
        \\ \cline{2-6} 
                                        & \multirow{2}{*}{\textrm{Mean}}         & \textrm{Packing}   & $21.376$ & $0.668$ & $32.0 \times$
        \\ \cline{3-6} 
                                        &                                        & \textrm{Unpacking} & $62.058$ & $3.095$ & $20.1 \times$
        \\ \cline{2-6} 
                                        & \multirow{2}{*}{\textrm{Gaussian}}     & \textrm{Packing}   & $38.003$ & $1.200$ & $31.7 \times$
        \\ \cline{3-6} 
                                        &                                        & \textrm{Unpacking} & $136.008$ & $7.108$ & $19.1 \times$
        \\ \hline
        \multirow{6}{*}{\textrm{MeKKS}} & \multirow{2}{*}{\textrm{Differencing}} & \textrm{Packing}   & $104.529$ & $3.273$ & $31.9 \times$
        \\ \cline{3-6} 
                                        &                                        & \textrm{Unpacking} & $51.076$ & $1.589$ & $32.1 \times$
        \\ \cline{2-6} 
                                        & \multirow{2}{*}{\textrm{Mean}}         & \textrm{Packing}   & $104.604$ & $3.274$ & $31.9 \times$
        \\ \cline{3-6} 
                                        &                                        & \textrm{Unpacking} & $51.202$ & $1.590$ & $32.2 \times$
        \\ \cline{2-6} 
                                        & \multirow{2}{*}{\textrm{Gaussian}}     & \textrm{Packing}   & $104.884$ & $3.277$ & $32.0 \times$
        \\ \cline{3-6} 
                                        &                                        & \textrm{Unpacking} & $51.319$ & $1.596$ & $32.2 \times$
        \\ \hline
    \end{tabular}
    }
    \caption[Packing and Unpacking Improvements]{Packing and unpacking times for each encryption scheme and inference algorithm, and the improvement gained.}
    \label{tab:packingAndUnpacking}
\end{table}

\setlength{\leftskip}{0cm}
\subsubsection{Transmission Times}
\setlength{\leftskip}{0.5cm}
\indent \indent
The other key aspect of the network component of the project is transmitting the data. One fundamental flaw of HE is the memory consumption inflation caused by encrypting data. Consequently, transmission times are much slower than when working with plain video data. In the final application, two main techniques were used to reduce this impact: vectorisation and compression.
\smallskip \\ \indent
For compression, several algorithms were tested - they were compared for both running time and compression ratio - and the algorithm that performed best on CKKS data was selected. The results of these tests are included in Table \ref{tab:compression}. From this, the impact of compressing videos is shown in Figure \ref{fig:compression1}.
\smallskip \\ \indent
Already, this provides good improvements over raw data. However, this can be extended by encrypting rows of video frames as a single ciphertext rather than each pixel distinctly. Figure \ref{fig:compression2} depicts the results of this adaptation.
\smallskip \\ \indent
As a result of the above optimisations, the running times for the client and server are summarised by Figure \ref{fig:clientTimeGraph} and Figure \ref{fig:serverTimeGraph} respectively.

\begin{table}
    \centering
    \def\arraystretch{1.25}
    % \resizebox{\textwidth}{!}{%
    \begin{tabular}{|c||c|c|c|c|}
        \hline
        \textrm{\textbf{Algorithm}} & \textrm{\textbf{Compression} (s)} & \textrm{\textbf{Decompression} (s)} & \textrm{\textbf{Size} (KB)} & \textrm{\textbf{Percentage}}
        \\ \hline \hline
        \textrm{None} & - & - & $716.18$ & $100\%$
        \\ \hline
        \texttt{gzip} & $94.69$ & $3.55$ & $442.46$ & $61.78\%$
        \\ \hline
        \texttt{bz2} & $36.28$ & $5.02$ & $435.29$ & $60.78\%$
        \\ \hline
        \texttt{lzma} & $294.85$ & $20.98$ & $421.86$ & $58.9\%$
        \\ \hline
        \texttt{brotli} & $871.97$ & $0$ & $435.78$  & $60.85\%$
        \\ \hline
    \end{tabular}
    % }
    \caption[Compression algorithms]{Evaluation of compression algorithms.}
    \label{tab:compression}
\end{table}

\begin{figure}
    \centering
    \begin{subfigure}[b]{0.495\textwidth}
        \centering
        \includegraphics[width=\textwidth]{figures/naiveMemUsageCKKS.eps}
        \caption{CKKS}
    \end{subfigure}
    \hfill
    \begin{subfigure}[b]{0.495\textwidth}
        \centering
        \includegraphics[width=\textwidth]{figures/naiveMemUsageMeKKS.eps}
        \caption{MeKKS}
    \end{subfigure}
    \caption{Na\"ive Memory Usage Compression Comparison}
    \label{fig:compression1}
\end{figure}

\begin{figure}
    \centering
    \begin{subfigure}[b]{0.495\textwidth}
        \centering
        \includegraphics[width=\textwidth]{figures/memUsageCKKS.eps}
        \caption{CKKS}
    \end{subfigure}
    \hfill
    \begin{subfigure}[b]{0.495\textwidth}
        \centering
        \includegraphics[width=\textwidth]{figures/memUsageMeKKS.eps}
        \caption{MeKKS}
    \end{subfigure}
    \caption{Vectorised Memory Usage Compression Comparison}
    \label{fig:compression2}
\end{figure}

\begin{figure}
    \centering
    \begin{subfigure}[t]{0.495\textwidth}
        \centering
        \includegraphics[width=\textwidth]{figures/clientTimes.eps}
        \caption{Client Running Times}
        \label{fig:clientTimeGraph}
    \end{subfigure}
    \hfill
    \begin{subfigure}[t]{0.495\textwidth}
        \centering
        \includegraphics[width=\textwidth]{figures/serverTimes.eps}
        \caption{Server Running Times}
        \label{fig:serverTimeGraph}
    \end{subfigure}
    \caption{Client and Server Running Times}
    \label{fig:clientAndServerGraph}
\end{figure}

\setlength{\leftskip}{0cm}

\subsection{Inference}
\setlength{\leftskip}{0.5cm}
\indent \indent
Another area of investigation that must be considered when discussing practicality is the performance of inference algorithms. This can be approached from two metrics. Firstly, the \textit{running time} must be considered to evaluate if algorithms will be able to return results in a reasonable amount of time. Secondly, \textit{accuracy} must be analysed to assess the quality of inference results compared to plain inference.

\setlength{\leftskip}{0cm}
\subsubsection{Running Time}
\setlength{\leftskip}{0.5cm}
\indent \indent
The running time for each inference algorithm varies significantly and is severely impacted by the parameters used to tune the accuracy of each algorithm, as discussed in §\ref{sec:adaptations}. Therefore, for this comparison, the parameters were tuned using the CKKS scheme and kept constant for a fair evaluation when testing the MeKKS scheme. The results are depicted by Figure \ref{fig:inferenceTime}. As expected, the CKKS scheme runs much quicker than the MeKKS scheme due to the more optimised implementation.

\begin{figure}
    \centering
    \begin{subfigure}[b]{0.495\textwidth}
        \centering
        \includegraphics[width=\textwidth]{figures/inferenceTimesCKKS.eps}
        \caption{CKKS}
    \end{subfigure}
    \hfill
    \begin{subfigure}[b]{0.495\textwidth}
        \centering
        \includegraphics[width=\textwidth]{figures/inferenceTimesMeKKS.eps}
        \caption{MeKKS}
    \end{subfigure}
    \caption{Inference Times}
    \label{fig:inferenceTime}
\end{figure}

\setlength{\leftskip}{0cm}
\subsubsection{Accuracy}
\setlength{\leftskip}{0.5cm}
\indent \indent
The accuracies of each HE inference algorithm are compared in Figure \ref{fig:accuracy}. The Moving-MNIST dataset allows accuracy to be easily calculated because it only contains white moving objects on a black background. Therefore, the similarity between the inference result and the original video can be determined by calculating the \textit{sum square difference} according to Equation \ref{eq:sumSquareDiff}. From this, the accuracy of each encryption scheme can be compared for each inference method.

\begin{figure}
    \centering
    \includegraphics[scale=0.8]{figures/accuracy.eps}
    \caption{Inference Accuracy}
    \label{fig:accuracy}
\end{figure}

\begin{equation}
    \label{eq:sumSquareDiff}
    \begin{split}
    &S_{sq} = \sum_{n,m \in N^{N \times M}} (J[n, m] - I[n, m])^2 \\
    &\textrm{which can be normalised using} \\
    &\frac{S_{sq}}{\sqrt{\sum J[n,m]^2 \times \sum I[n,m]^2}}
    \end{split}
\end{equation}
given two images $J[x,y]$ and $I[x,y]$ with $(x,y) \in N^{N \times M}$.

\setlength{\leftskip}{0cm}




\section{Summary}
\setlength{\leftskip}{0.25cm}
\indent \indent
This evaluation provides evidence that combining the domains of homomorphic encryption and moving object detection can produce promising results comparable to existing algorithms extracting moving objects from unencrypted, plain video data. Moreover, it demonstrates that progress can be made in improving the performance of systems incorporating these techniques. However, it acknowledges that further research is required into homomorphic primitives to provide more operations on encrypted data and reduce the space complexity of encrypted data to improve the time complexity of data transmission and inference algorithms.

\setlength{\leftskip}{0cm}

\chapter{Conclusions}
\label{chap:conclusions}

\section{Project Summary}
\indent \indent
Overall, the project was a success. All core success criteria and one of the extensions were met, and the remaining extensions were thoroughly considered. In its final state, the project contains a novel investigation into the applicability of HE in ensuring user privacy is preserved when using modern MLaaS surveillance technology. For this, a client-server system allowing data to be homomorphically encrypted and transferred across a network was implemented. Moreover, several background subtraction algorithms were implemented, and more were investigated to attempt to find the limit of HE pertinency. Also, the results of these algorithms were evaluated against traditional algorithms operating on plain data, revealing similar accuracy but significantly increased running time due to increased computational complexity. Similarly, network activity was compared between plain and encrypted data, highlighting the cost of increased ciphertext size through increased data manipulation running times and transmission times between client and server.   Furthermore, a bespoke implementation of the CKKS scheme - called MeKKS - was created to increase understanding and highlight opportunities for application-specific optimisations.







\section{Lessons Learned}
\indent \indent
Throughout the project, many challenges were faced. The project investigated a novel integration of HE and unsupervised machine learning for image segmentation - neither of which are covered by Tripos content. Consequently, the most critical challenge was the need for research to cultivate background knowledge. However, there was little overlap between these topics found in published work, so significant time had to be invested in deriving potential solutions to encountered problems. Specifically concerning HE, lack of documentation regarding scheme applications warranted many hours of literature reviews and deciphering research papers. 
\smallskip \\ \indent
Furthermore, computations using machine learning and HE took a long time to execute - particularly when networking was involved. Consequently, debugging and evolving algorithms was a time-consuming process. This made time management even more critical to allow enough time to complete a thorough investigation. However, these challenges did make the project a productive learning experience that will be helpful when exploring other areas of computer science in future or when working on any large projects generally.







\section{Future Directions}
\indent \indent
The project encountered two recurring limitations of HE that would benefit from further investigation. Firstly, the memory requirements resulting from ciphertext size resulted in a significant bottleneck during the networking stage of the application. Consequently, research into reducing this cost would make HE more applicable to MLaaS applications. Secondly, the variety of operations available in the HE domain must be expanded if more advanced moving object detection algorithms are to be implemented. This would enable GMMs to produce much more accurate inference results.
\smallskip \\ \indent
Regarding security, other methods could be investigated to preserve privacy. For example, techniques exist for storing private keys in hardware to prevent visibility to users ~\cite{Lorch}. This would reduce the risk of malicious actors performing unauthorised access. Moreover, \textit{functional cryptography} is an asymmetric cryptographic protocol that allows properties of ciphertexts to be extracted ~\cite{Boneh}. An investigation into this could produce a potential alternative to HE, and compare the advantages and disadvantages of the two approaches.
\smallskip \\ \indent
More specifically for surveillance applications, further research considering the hardware aspect of the problem would be beneficial for obtaining a more precise measure of practicality. Some potential topics include: evaluating and reducing energy usage to accommodate battery-powered devices, improving computational complexity for execution on lower-powered processors, or developing specialised accelerators to design devices specific to HE computations. 

%\include{3Introduction_papers}  % Introduction to the papers
%\include{4Papers}               % The papers, with cover pages
%


%--------------------------------------------------------------------------------------------------
% BACK MATTER
%--------------------------------------------------------------------------------------------------
{
	\backmatter
	
	%
	% The bibliography
	%
	
	
	\Urlmuskip=0mu plus 1mu\relax
	\printbibliography[title=Bibliography,heading=bibintoc]
}
	
% Appendices
%
\appendix
\chapter{Project Proposal}
\label{app:proposal}

\indent \indent
The original proposal for the project is included from the next page.

\includepdf[pages=-]{originalProposal.pdf}
             % Appendix A: Something

\end{document}

